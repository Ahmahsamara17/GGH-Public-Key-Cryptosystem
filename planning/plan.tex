\documentclass{article}
\usepackage{amsmath}
\usepackage{parskip}
\usepackage{amsfonts}
\usepackage{relsize}
\begin{document}
- GGH Public Key Cryptosystem\newline

	==== Key Creation ====\newline

1. Choose a good basis $v_1, \text{...}, v_n$.

$\rightarrow$ Choose a set of lineraly independant vectors:
$$v_1, v_2, \text{...}, v_n \in \mathbb{Z}^n $$

- We need them to be reasonably orthogonal. The method I will use is to fix a paramter d and pick random coordinates between $[-d, d]$, we then check how orthogonal they are by computing the Hadamard ratio. 

- Let $V$ be the $n-\text{by}-n$ matrix whose rows are the the vecrots $v_1, \text{...}, v_n$ and let $L$ be the lattice generated by those vectors.

2. Choose an integer matrix U satisfying $det(U) = ±1$.

$\rightarrow$ To create $U$, we take a product of a large number of randomly chosen elementary matrices

3. Compute a bad basis $w_1, \text{...}, w_n$ as the rows of $W = UV$.

4. Publish the public key $w_1, \text{...}, w_n$.\newline


	==== Encryption ====\newline

1. Choose small plaintext vector \textit{m}.

2. Choose random small vector \textit{r}.

$\rightarrow$ Choose r between $[-\delta, \delta]$, where $\delta$ is a fixed public paramter.

3. Use Alice’s public key to compute $e = x_1v_1 + \text{···} + x_nv_n + r.$

4. Send the ciphtertext \textit{e} to Alice.\newline

$$ e = mW + r = \sum^{n}_{i=1}m_iw_i + r $$
	==== Decryption ====\newline

1. Use Babai’s algorithm to compute the vector $v \in L$ closest to \textit{e}.

2. Compute $vW^{-1}$ to recover \textit{m}.

\end{document}

